%%%%%%%%%%%%%%%%%%%%%%%%%%%%%%%%%%%%%%%%%%%%%%%%%%%%%%%%%%%%%%%%%%%
%%%%% Основной файл для авторов журнала MICME
%%%%% факультет математики и информатики
%%%%% Таврическая академия Крымского федерального университета им. В.И. Вернадского
%%%%% e-mail: micme2017@yandex.ru
%%%%% site: http://micme.cfuv.ru/
%%%%%%%%%%%%%%%%%%%%%%%%%%%%%%%%%%%%%%%%%%%%%%%%%%%%%%%%%%%%%%%%%%%

%ПРАВИЛА ОФОРМЛЕНИЯ МАТЕРИАЛОВ:
%
%Материалы должны быть оформлены с использованием стилевого файла micmo.cls по образцу примера MICME.tex в редакторе LaTeX на русском или английском языке.
%
%Название файла формируйте следующим образом Surname.tex
%Surname – фамилия, написанная по-английски без пробелов.



\documentclass{micmo}
\usepackage[cp1251]{inputenc}
\usepackage[T2A]{fontenc}
\usepackage[english,ukrainian,russian]{babel}
\sloppy
%\newcommand{\ds}{\displaystyle}

\topmatter{В.\;С.\;Козенко} %укажите свои инициалы и фамилию в предложенном формате

\udc{XXX.XX} %укажите УДК вашей работы

\title{Сравнительный анализ библиотек глубокого обучения} %укажите название вашей работы

%\thanks{Работа выполнена при финансовой поддержке~\ldots}

\author{В.\;С.\;Козенко}%\author{инициалы и фамилия}
{Крымский федеральный университет им.\;В.\;И.\;Вернадского, \textit{e-mail: mail@.ru}} %{контактные данные}

\date{01.04.2020} %дата подачи работы

\enabstracts{XXAXX}%\enabstracts{обозначения MSC2010}
{V.\;S.\;Kozenko}%{инициалы и фамилия на английском языке}
{V.\;I.\;Vernadsky Crimean Federal University} %{контактные данные на английском языке}
{English title}%{название работы на английском языке}
{\ldots.}%{аннотация на английском языке}
{\ldots.}%{ключевые слова на английском языке}

\begin{document}
\begin{article}

\abstracts{~\ldots.}%{текст аннотации}
{\ldots.} %{ключевые слова}


%\section*{Введение}

%В классической литературе~\cite[c. 45]{Surname:year:1}

\section{Сравнение библиотек глубокого обучения}
\subsection{Теоритическое сравнение}

TensorFlow - открытая библиотека для машинного обучения, разработанная компанией Google, для решения задач построения и тренировки нейронных сетей. Основной API для работы с библиотекой реализован для Python также существуют реализации для C Sharp, C++, Haskell, Java, Go и Swift. Является продолжением закрытого проекта DistBelief. Изначально TensorFlow была разработана командой Google Brain для внутреннего использования в Google, в 2015 году система была переведена в свободный доступ [7].
PyTorch - библиотека машинного обучения для языка Python с открытым исходным кодом, созданная на базе Torch. Разрабатывается преимущественно группой искусственного интеллекта Facebook. Также вокруг этого фреймворка выстроена экосистема, состоящая из различных библиотек, разрабатываемых сторонними командами: Fast.ai, упрощающая процесс обучения моделей, Pyro, модуль для вероятностного программирования, от Uber, Flair, для обработки естественного языка и Catalyst, для обучения DL и RL моделей [7].
Keras - открытая нейросетевая библиотека, написанная на языке Python. Она представляет собой надстройку над фреймворками Deeplearning4j, TensorFlow и Theano. Нацелена на оперативную работу с сетями глубинного обучения, при этом спроектирована так, чтобы быть компактной, модульной и расширяемой. Она была создана как часть исследовательских усилий проекта ONEIROS (Open-ended Neuro-Electronic Intelligent Robot Operating System), а ее основным автором и поддерживающим является Франсуа Шолле (François Chollet), инженер Google. Планировалось что Google будет поддерживать Keras в основной библиотеке TensorFlow, однако Шолле выделил Keras в отдельную надстройку, так как согласно концепции Keras является скорее интерфейсом, чем сквозной системой машинного обучения. Keras предоставляет высокоуровневый, более интуитивный набор абстракций, который делает простым формирование нейронных сетей, независимо от используемой в качестве вычислительного бэкенда библиотеки научных вычислений. Microsoft работает над добавлением к Keras и низкоуровневых библиотек CNTK [7].



%\begin{figure*}[h!]
%\centering{\includegraphics[scale=0.3]{Images/icon.eps}}
%\caption{Вставка рисунка в формате eps.}
%\end{figure*}

%Включать рисунки

%Рассмотрим уравнение
%\begin{equation}\label{eq:1}
%\int_{-\infty}^{\infty}~\ldots
%\end{equation}

%Согласно~\eqref{eq:1} введ\"ем

%\begin{definition}\label{def:1}

%\end{definition}

%Учитывая определение~\ref{def:1}, получим следующую
%\begin{theorem}\label{th:1}
%~\ldots
%\begin{proof}
%~\ldots
%\end{proof}
%\end{theorem}

%\begin{corollary}\label{cor:1}
%~\ldots
%\begin{proof}
%~\ldots
%\end{proof}
%\end{corollary}

\section*{Заключение}
%Основным результатом данной работы является теорема~\ref{th:1}. Из следствия~\ref{cor:1}~\ldots

%\newpage

%источник задаётся в формате, соответствующем ГОСТ Р 7.0.9 – 2009 («Система стандартов по информации, библиотечному и издательскому делу. Библиографическое обеспечение издательских и книготорговых процессов. Общие требования»)
%чтобы использовать цитирование в тексте, сделайте в списке источников указание вида:
%\refitem{название ссылки для цитирования}
%{имя автора}

%\begin{references}
%\refitem{Surname:year:1}
%{Фамилия\;И.\;О.} Название публикации~// Журнал.~-- Год.~-- том X, вып. X~-- С.~1--2.



%\end{references}

%\listofruabstracts\listofenabstracts

\end{article}
\end{document}