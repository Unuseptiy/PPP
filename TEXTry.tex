\documentclass[a4paper]{article}
%\usepackage[cp1251]{inputenc} % Выбор кодировки русского языка
\usepackage[utf8]{inputenc} % Выбор кодировки русского языка
\usepackage[russian]{babel}   % Пакет поддержки русского языка
\usepackage{graphicx}         % Пакет для вставки рисунков
\usepackage[12pt]{extsizes}

\usepackage{amsmath}
\usepackage{amsfonts,amssymb}
\usepackage{amsthm}
\usepackage[active]{srcltx}
\sloppy
\def\baselinestretch{1}						

\begin{document}
%\maketitle
%\section{}
%\subsection{}

правилу $MP$. Тогда $F_j = F_i \rightarrow F_k$ (или $F_i = F_j \rightarrow F_k$). Так как $i,\ j < k$, по предположению индукции имеем равенства $\varphi (F_i) = 1$ и $\varphi (F_j) = \varphi(F_i \rightarrow F_k) = 1$. Но тогда очевидно, что $\varphi(F_k ) = 1$.

Осталось рассмотреть случай, когда $F_k$ получается из $F_i$ по правилу обобщения. В этом случае $F_i = F_i(x)$ и $F_k = (\forall y)F_i(y)$.
Равенство
$$
\varphi[(\forall y)F_i(y)] = 1,
$$
которое надо доказать, означает, что для любой интерпретации $\varphi' \in I(\varphi, y)$ выполняется равенство $\varphi' (F_i(y)) = 1$. Возьмем $\varphi'$ из $I(\varphi, y)$. Рассмотрим интерпретацию $\psi \in I(\varphi, x)$ такую, что $\psi(x) = \varphi'(y)$. Так же, как и раньше, доказываем, что $\psi (F_i(x)) = 1$. По теореме 2 получаем равенство $\psi(F_i(x)) = \varphi' (F_i(y))$. Следовательно, $\varphi' (F_i(y)) = 1$.

\subsection{\S6 Теорема о непротиворечивости (леммы)}

Начнем с основного определения.

\textbf{Определение.} Множество формул $P$ называется противоречием, если существует формула $F$ такая, что $P \vdash F$ $P \vdash \neg F$.

\textbf{Пример 9.} Убедимся в том, что множество формул 
$$
	P = \{(\forall y)(F(y) \rightarrow G(a, b)), F(a), (\forall x)\neg G(a, x)\}
$$
является противоречивым. Рассмотрим последовательности формул $D_1, \dots, D_5$ и $E_1, E_2, E_3$:

$D_1 = (\forall y)(F(y) \rightarrow G(a, b)) \rightarrow (F(a) \rightarrow G(a, b)),$

$D_2 = (\forall y)(F(y) \rightarrow G(a, b)),$

$D_3 = F(a) \rightarrow G(a, b), $

$D_4 = F(a),$

$D_5 = G(a ,b),$

$E_1 = (\forall x) \neg G(a, x),$

$E_2 = (\forall x) \neg G(a, x) \rightarrow \neg G(a, b),$

$E_3 = \neg G(a, b).$

Легко видеть, что эти последовательности являются выводами из $P$. Следовательно, $P \vdash G(a, b)$ и $P \vdash \neg G(a, b)$. Это означает, что множество $P$ противоречиво.\ $\square$


Укажем два полезных для дальнейшего свойства противоречивых (и непротиворечивых) множеств формул.

\textbf {Свойство 1.} Из примера 2 легко следует, что \textit{противоречивое множество можно было определить как множество, из которого выводима любая формула}.

\textbf{Свойство 2.} Если из множества формул $P$ не выводима некоторая формула $G$,  то множество $P \cup \{\neg G\}$ непротиворечиво.

Действительно, пусть найдется формула $F$ такая, что $P~\cup~\{ \neg G\}~\vdash~F$ и $P~\cup~\{ \neg G \}~\vdash~\neg F$. По теореме о дедукции получаем, что 
$$
P \vdash \neg G \rightarrow F и P \vdash \neg G \rightarrow \neg F
$$

Из этих утверждений и аксиомы по схеме АЗ $$(\neg G \rightarrow \neg F)\rightarrow ((\neg G \rightarrow F) \rightarrow G)$$ следует, что $P \vdash G$. Получили противоречие с условием. Следовательно, множество формул $P \cup \{\neg G\}$ непротиворечиво.

Целью данного и следующего параграфов является доказательство утверждения о том, что если множество формул непротиворечиво, то оно выполнимо. Это утверждение называется
теоремой о непротиворечивости. При доказательстве этой теоремы нам будет удобно считать, что непротиворечивое множество формул $P$ удовлетворяет следующим двум формальным условиям:

($\alpha$) для любой формулы $F$ сигнатуры множества $P$ либо $F \in P$, либо $\neg F \in P$; 

($\beta$) если множество $P$ содержит формулу $\neg(\forall x)F(x)$, то найдется переменная $w$, которой нет в этой формуле, такая, что $\neg F(w) \in P$.

Хотя условия ($\alpha$) и ($\beta$) были названы формальными, им можно придать некоторый содержательный смысл. Действительно, условие ($\alpha$) означает, что множество P является максимальным среди всех непротиворечивых множеств данной сигнатуры. (Конечно, пока неясно, что такое множество существует.) Смысл условия ($\beta$) состоит в следующем. Формула $\neg (\forall x)F(x)$ в логике с обычным набором связок и кванторов равносильна формуле $(\exists x)\neg F(x)$. Условие ($\beta$) требует, чтобы утверждение о  существовании элемента x, для которого формула F(x) ложна, «реализовалось» на некоторой переменной w.  

Данный параграф посвящен «обеспечению» этих условий. Доказательство самой теоремы о непротиворечивости будет изложено в следующем параграфе.

\textbf{Лемма 1.} Пусть $P$ - непротиворечивое множество формул сигнатуры $\Sigma$. Тогда существует множество формул $P'$ той же сигнатуры $\Sigma$, удовлетворяющее условиям:
\begin{enumerate}
	\item $P \subseteq P'$,
	\item $P'$ непротиворечиво,
	\item $P'$ удовлетворяет условию ($\alpha$).
\end{enumerate}

Другими словами, всякое непротиворечивое множество формул содержится в максимальном непротиворечивом расширении.

\textit {Доказательство} леммы 1 использует лемму Цорна. Напомним ее формулировку: если в частично упорядоченном множестве $S$ каждая цепь имеет верхнюю грань, то множество $S$ содержит максимальный элемент. Цепь – это непустое линейно упорядоченное подмножество множества $S$.

Пусть $S$ – множество всех непротиворечивых расширений множества $P$, имеющих сигнатуру $\Sigma$, т. е.

$S = \{Q\ |\ Q$ имеет сигнатуру $\Sigma, P \subseteq Q$ и $Q$ непротиворечиво\}.

Ясно, что множество $S$ непусто, так как $P \in S$, что оно частично упорядочено отношением теоретико-множественного включения. Возьмем в S цепь  $$
T  = \{ Q_i\ |\ i \in I \}.
$$

В качестве "кандидатуры" на максимальную грань цепи рассмотрим множество 
$$
Q' = \cup \{ Q_i\ |\ i \in I \}.
$$

Надо убедиться в том, что $Q' \in S$. Очевидно, что $Q'$ имеет сигнатуру $\Sigma$ и что $P \subseteq Q'$. Предположим, что $Q'$ противоречиво. Тогда существует формула $F$ такая, что $Q' \vdash F$ и $Q' \vdash \neg F$. Рассмотрим вначале первый случай, т. е. что $Q' \vdash F$. Пусть 
\begin{align}
	& H_1, H_2, \dots, H_n\ - \tag{*}
\end{align}
вывод формулы $F$ из множества $Q'$. В выводе выберем все гипотезы. Пусть это будут формулы 
$$
H_{i1},\ H_{i2},\ \dots,\ H_{ik}.
$$

Тогда в  цепи $T$ найдутся множества $Q_{i1},\ Q_{i2},\ \dots,\ Q_{ik}$ такие, что $H_{i1} \in Q_{i1},\ H_{i2} \in Q_{i2},\ \dots,\ H_{ik} \in Q_{ik}.$ Эти множества являются элементами цепи, поэтому они попарно сравнимы относительно теоретико-множественного включения. Следовательно, среди них есть наибольшее по включению множество. Пусть это будет $Q_{ik}$ . Тогда вывод (*) будет выводом из $Q_{ik}$ , т. е. $Q_{ik} \vdash F$. Аналогично доказывается существование элемента цепи $Q_{il}$ такого, что $Q_{il} \vdash \neg F$. Воспользуемся еще раз тем, что $Q_{ik}$ и $Q_{il}$ – элементы цепи, и потому среди них есть наибольший. Пусть это будет $Q_{il}$. Но тогда $Q_{il} \vdash F$ и $Q_{il} \vdash \neg F$. Это противоречит тому, что $Q_{il} \in S$. Следовательно, $Q'$ непротиворечиво, и поэтому $Q' \in S$. Условие леммы Цорна выполнено.

Заключение леммы Цорна говорит о том, что $S$ содержит (хотя бы один) максимальный элемент. Обозначим его через $P'$. Тот факт, что $P \subseteq P'$ и что $P'$ – непротиворечиво следует из принадлежности $P' \in S$. Осталось проверить, что $P'$ удовлетворяет условию ($\alpha$).  Предположим противное, пусть существует формула $F$ такая, что $F \notin P'$ и $\neg F \notin P'$. Тогда по свойству 2 множества формул $P' \cup \{F\}$ и $P' \cup \{\neg F\}$ противоречивы, так как $P'$ – максимальное непротиворечивое множество. Проанализируем противоречивость множества $P' \cup \{\neg F\}$. По определению противоречивого множества существует формула $G$ такая, что
$$
P' \vdash G,\ P' \vdash \neg G %здесь вместо запятой как-то надо букву "и" вставить
$$

Как уже неоднократно делалось, применение здесь теоремы о дедукции и аксиомы по схеме 3 дает, что $P' \vdash F$. Но тогда множество $P' \in \{F\}$ не может быть противоречивым, так как $P' \in \{F\} = P'$ и $P'$ непротиворечиво. Следовательно, $P'$ удовлетворяет условиям 1) – 3) из формулировки леммы 1. $\square$


С «обеспечением» условия ($\beta$) дело обстоит несколько сложнее. Мы вначале в лемме 2 получим ослабленный вариант этого условия.

\textbf{Лемма 2}. Пусть $P$ – непротиворечивое множество формул сигнатуры $\Sigma$. Тогда существует сигнатура $\Sigma'$ и множество формул $P'$ сигнатуры $\Sigma'$, для которого выполняются условия:
% здесь маркеры можно как-то по-другому задать, вероятно
\begin{itemize}
	\item [$1)$] $P \subseteq P'$,
	\item [$2)$] $P'$ непротиворечиво,
	\item [$3)$] если формула $\neg (\forall x)F(x)$ принадлежит $P$, то найдется новая переменная $w$ такая, что формула $\neg F(w)$ принадлежит $P'$.
\end{itemize}

\textit{Доказательство.} Пусть
$$
M = \{ \neg (\forall x_i)F_i(x_i) | i \in I \} \ -
$$
множество всех формул вида $\neg (\forall x)F(x)$, содержащихся в P. Для каждой такой формулы $\neg (\forall x_i)F_i(x_i)$ из $M$ добавим к $\Sigma$ новую (такую, которой нет в $P$) переменную $w_i$. Получим сигнатуру $\Sigma'$. В качестве $P'$ рассмотрим множество
$$
P' = P \cup \{ \neg F_i(w_i) | i \in I \}.
$$


Ясно, что $P'$ удовлетворяет условиям 1) и 3) леммы 2. Надо доказать, что $P'$ непротиворечиво. Предположим противное: пусть множество $P'$ противоречиво. Тогда достаточно рассмотреть множество $P'$, полученное из $P'$ добавлением конечного множества формул вида $\neg (\forall x)F(x)$, а следовательно, и одной формулы такого вида. 

Итак, пусть $\neg (\forall x)F(x) \in P, P' = P \cup \{\neg F(w)\}$ и множество $P'$ противоречиво. По определению противоречивости существует формула $G$ такая, что
$$
P \cup \{ \neg F(w) \} \vdash G,\ p \cup \{ \neg F(w) \} \vdash \neg G. %та же ситуация с "и".
$$

Тогда, как это часто демонстрировалось, $P \vdash F(w)$. Применение правила обобщения дает, что $P \vdash (\forall x)F(x)$. И в то же время, $\neg (\forall x)F(x) \in P$. Получили противоречие с условием о непротиворечивости множества $P$. Итак, множество $P'$ удовлетворяет условиям 1) -- 3) из формулировки леммы.

\textbf{Лемма 3.} Для любого непротиворечивого множества $P$ существует непротиворечивое расширение $P'$, удовлетворяющее условиям ($\alpha$) и ($\beta$).

\textit{Доказательство.} Построим последовательность множеств формул $Q_0,\ Q_1,\ \dots,\ Q_n,\ \dots$ сигнатур $\Sigma_0,\ \Sigma_1,\ \dots, \Sigma_n,\ \dots$ соответственно следующим образом. Пусть $Q_0 = P$ и $\Sigma_0$ -- сигнатура множества формул $P$. К множеству $Q_0$ применим лемму 1, получим непротиворечивое множество $Q_1$, удовлетворяющее условию ($\alpha$), сигнатуры $\Sigma_1$, равной $\Sigma_0$. Далее к множеству $Q_1$ применим лемму 2, получим непротиворечивое множество $Q_2$ сигнатуры $\Sigma_2$. Затем снова применим лемму 1, но теперь уже к множеству $Q_2$ и т. д. В итоге возникнет последовательность непротиворечивых расширений множества $P$:
$$
P = Q_0 \subseteq Q_1 \subseteq \dots \subseteq Q_n \subseteq \dots
$$


Каждое множество с четным номером удовлетворяет условию ($\alpha$). А каждое множество с нечетным номером удовлетворяет условию 3) леммы 2, более слабому, чем условие ($\beta$). Для удобства изложения повторим третье условие леммы 2 в новых обозначениях: если $\neg (\forall x)F(x) \in Q_l$ и $l$ -- нечетно, то существует новая переменная $w$ такая, что $\neg F(w) \in Q_{l+1}$. Условие 3 в такой «редакции» обозначим через (4).

В качестве $P'$ рассмотрим объединение построенной последовательности множеств формул:
$$
P' = Q_0 \cup Q_1 \cup \dots \cup Q_n \cup \dots\ .
$$
а в качестве - объединение сигнатур:
$$
\Sigma' = \Sigma_0 \cup \Sigma_1 \cup \dots \cup \Sigma_n \cup \dots\ .
$$


Доказательство непротиворечивости множества $P'$ в близкой ситуации (см. проверку условия леммы Цорна) мы уже проводили. Действительно, если найдется формула $F$ такая, что $P' \vdash F$ и $P' \vdash \neg F$ то существует множество $Q_n$ с тем же свойством. Это противоречит тому, что $Q_n$ непротиворечиво.

Докажем, что множество $P'$ удовлетворяет условию ($\alpha$). Пусть $F$ -- формула сигнатуры $\Sigma'$. Так как $\Sigma'$ -- объединение возрастающей последовательности сигнатур существует сигнатура $\Sigma_k$ такая, что $F$ является формулой сигнатуры $\Sigma_k$ . Можно считать, что $k$ -- четное число (иначе вместо $n$ можно взять $k + 1$). Но тогда $Q_{k+1}$ удовлетворяет условию ($\alpha$) и поэтому либо $F \in  Q_{k+1}$, либо $\neg F \in Q_{k+1}$. По построению множества $P'$ отсюда следует, что либо $F \in P'$, либо $\neg F \in P'$. 

Установим, что множество $P'$ удовлетворяет условию ($\beta$). Возьмем формулу $\neg (\forall x)F(x)$ из $P'$. Тогда найдется множество $Q_l$, содержащее эту формулу. Можно считать, что $l$ -- нечетное число (иначе вместо $l$ можно взять $l + 1$). Тогда по условию (4) существует переменная $w$ такая, что $\neg F(w) \in Q_{l+1}$. Так как $Q_{l+1}$ содержится в $P'$, получаем, что $\neg F(w) \in P'$. $\square$

\subsection{\S7 Теорема о непротиворечивости (доказательство теоремы)}


Напомним вначале формулировку теоремы о непротиворечивости.
\textbf{Теорема 3.5.} Если множество формул $P$ непротиворечиво, то $P$ выполнимо, т. е. существует интерпретация $\varphi$ такая, что  $\varphi (P) = 1$.

\textit{Доказательство.} В силу леммы 3 можно считать, что $P$ удовлетворяет условиям ($\alpha$) и ($\beta$).

Сигнатуру множества $P$ обозначим через $\Sigma$. В качестве области интерпретации возьмем множество термов $T$ сигнатуры $\Sigma$. Функцию $\varphi$ определим следующим образом:
\begin{itemize}
	\item [$1)$] $\varphi(x) = x$,
	\item [$2)$] $\varphi(c) = c$,
	\item [$3)$] $(\varphi f)(t_1, t_2, \dots, t_n) = f(t_1, t_2, \dots, t_n)$,
	\item [$4)$] $(\varphi r)(t_1, t_2, \dots, t_n) = 1 \leftrightarrow r(t_1, t_2, \dots, t_n) \in P$. %тут со стрелкой чет
\end{itemize}


Здесь $x$ -- переменная, $c$ -- константа, $f$ и $r$ -- соответственно $n$-местные символы функции и предиката.

Легко видеть, что из первых трех пунктов определения интерпретации следует, что $\varphi(t) = t$ для любого терма t.

Индукцией по построению интерпретации докажем, что
\begin{align}
	& \varphi (F) = 1 \Leftrightarrow F \in P. \tag{*}
\end{align}


Из этой эквиваленции очевидным образом следует выполнимость множества формул $P$.

\textit{База индукции}: $F$ -- атомарная формула. Эквиваленция (*) следует из построения интерпретации (см. пункт 4), так как множество $P$ одновременно не может содержать атомарную формулу и ее отрицание.

\textit{Шаг индукции}: предположим, что для всех собственных подформул формулы $F$ эквиваленция (*) доказана. Рассмотрим три случая.

\textit{Случай 1}: $F = \neg G$. Тогда выполняются следующие эквиваленции:
$$
\varphi(F) = 1 \Leftrightarrow \varphi(\neg G) = 1 \Leftrightarrow \varphi(G) = 0 \Leftrightarrow G \notin P \Leftrightarrow \neg G \in P \Leftrightarrow F \in P.
$$

В дополнительных комментариях нуждаются только две эквиваленции: $\varphi(G) = 0 \Leftrightarrow G \notin P$ – по предположению индукции, $G \notin P \Leftrightarrow \neg G \in P$ – по условию ($\alpha$).

\textit{Случай 2}: $F = G \rightarrow H$. Этот случай сложнее первого. Доказательство эквиваленции (*) разобьем на две части: «туда» и «обратно». А каждую из этих частей разобьем еще на две возможности.

($\Rightarrow$) Пусть $\varphi (G \rightarrow H) = 1$.

\textit{Возможность 1}: $\varphi(H) = 1$. Тогда по предположению индукции $H \in P$, и $P \vdash G \rightarrow H$ (аксиома A1). Но в таком случае, $G \rightarrow H \in P$, так как $P$ непротиворечиво и удовлетворяет условию ($\alpha$).

\textit{Возможность 2}: $\varphi(H) = 0$. Так как $\varphi(G \rightarrow H) = 1$, имеем равенство $\varphi (G) = 0$. По предположению индукции отсюда следует, что $G \notin P$. Следовательно, $\neg G \in P$ и множество $P \cup \{G\}$ противоречиво. В таком случае, из $P \cup \{G\}$ выводима любая формула (свойство 2 и предыдущего параграфа), в том числе формула $H$. По теореме о дедукции получаем, что $P \vdash G \rightarrow H$. Следовательно, $G \rightarrow H \in P$.

($\Leftarrow$) Пусть $G \leftarrow H \in P$.
\textit{Возможность 1}: $H \in P$. Тогда по предположению индукции $\varphi (H) = 1$, и следовательно, $\varphi(G \rightarrow H) = 1$.

\textit{Возможность 2}: $H \notin P$. Тогда $G \notin P$, поскольку в противном случае из условий $G \in P$ и $G \rightarrow H \in P$ следует, что $H \in P$. По предположению индукции получаем равенство $\varphi(G) = 0$, и поэтому $\varphi(G \rightarrow H) = 1$.

Случай 2 рассмотрен полностью.

\textit{Случай 3}: $F = (\forall)G(x)$. Этот случай, как и предыдущий, разобьем на две части: «туда» и «обратно».

($\Rightarrow$) Пусть $\varphi[(\forall x)G(x)] = 1$. Надо доказать, что $(\forall x)G(x) \in P$.
Предположим противное: пусть $(\forall x)G(x) \notin P$. Тогда по свойствам ($\alpha$) и ($beta$) и найдется переменная $w$ такая, что $\neg G(w) \in P$. Еще раз воспользуемся свойством ($\alpha$), получим, что $G(w) \notin P$. По предположению индукции имеем равенство $\varphi(G(w)) = 0$. С другой стороны, известно, что аксиома $(\forall x)G(x) \rightarrow G(w)$ тождественно истинна. Посылка этой аксиомы истинна при интерпретации $\varphi$. Следовательно, $\varphi(G(w)) = 1$. Полученное противоречие показывает, что формула $(\forall x)G(x)$ принадлежит $P$.

%($\Leftarrow$)Пусть $(\forall x)G(x) \in P$. Надо доказать, что $\varphi[(\forall x)G(x)] = 1$. Как и в части «туда» здесь удобно рассуждать от противного. Предположим, что $\varphi [(\forall x)G(x)] = 0$. Тогда существует интерпретация $\varphi' \in I(\varphi, x)$ такая, что $\varphi' (G(x)) = 0$. Вспомним, наконец, что областью определения интерпретации $\varphi$ является множество термов $T$. Следовательно, $\varphi'(x)$ − это некоторый терм. Обозначим его через $t$.

Если подстановка $t$ вместо $x$ в формуле $G(x)$ допустима, то все заканчивается довольно просто. Действительно, тогда по теореме 2 получаем равенство $\varphi'(G(x)) = \varphi[G(t)] = 0$. По предположению индукции имеем, что $G(t) \notin P$. Получаем противоречие с предположением $(\forall x)G(x) \in P$ и аксиомой $(\forall x)G(x) \rightarrow G(t)$ по схеме A5.

Предположим теперь, что подстановка $t$ вместо $x$ в формуле $G(x)$ недопустима. Заменим в формуле $G(x)$ все связанные переменные так, чтобы эта подстановка была допустима. Полученную формулу обозначим через $\underline{G}(x)$. Если $(\forall x)\underline{G}(x) \in P$, то рассуждаем так же, как и в предыдущем абзаце. Пусть $(\forall x)\underline{G}(x) \notin P$. Тогда по условиям ($\alpha$) и ($\beta$) существует переменная $w$ такая, что $\neg \underline{G}(w) \in P$. По предположению индукции имеем равенство $\varphi (\underline{G}(w)) = 0$. Однако $\varphi (\underline{G}(w)) = \varphi (G(w))$ по теореме 1 о замене. Следовательно, $\varphi (G(w)) = 0$ и $G(w) \notin P$. Как и выше, получаем противоречие с предположением $(\forall x)G(x) \in P$ и аксиомой $(\forall x)G(x) \rightarrow G(w)$ по схеме A5. Случай 3 полностью рассмотрен. $\square$

\subsection{\S8 Теоремы о полноте и о компактности}


Цель этого параграфа доказать совпадение семантического понятия логического следования с синтаксическим понятием выводимости. Соответствующее утверждение, как уже отмечалось, называется теоремой о полноте. Кроме того, в этом параграфе будет доказана «обещанная» в предыдущей главе теорема о компактности.


\textbf{Теорема 3.6} (о полноте). Формула выводима из множества формул тогда и только тогда, когда она является логическим следствием этого множества.

\textit{Доказательство}. Необходимость уже доказана (см. теорему об оправданности аксиоматизации). Достаточность докажем методом от противного. Предположим, что формула $G$ является логическим следствием множества формул $P$ и $G$ не выводима из $P$. Тогда по свойству 2 непротиворечивых множеств (см. \S6) множество формул $P \cup \{\neg G\}$ непротиворечиво. Из теоремы о непротиворечивости следует, что оно выполнимо. Это означает, что существует интерпретация $\varphi$ такая, $\varphi (P) = 1$ и $\varphi (\neg G) = 1$, т.е. $\varphi (P) = 1$ и $\varphi(G) = 0$. Получили противоречие с тем, что формула $G$ является логическим следствием множества $P$. $\square$


\textbf{Теорема 3.7} (о компактности). Если формула $F$ является логическим следствием бесконечного множества формул $P$, то она является логическим следствием некоторого конечного подмножества $P_0$ множества $P$.

\textit{Доказательство}. Пусть формула $F$ является логическим следствием бесконечного множества формул $P$. Тогда по теореме о полноте формула $F$ выводима из $P$. Это означает, что существует вывод из $P$, последней формулой которого является формула $F$:
$$
F_1,\ F_2,\ \dots, F_n\ =\ F.
$$

Обозначим через $P_0$ множество тех формул из $P$, которые встречаются в выводе. Ясно, что $P_0$ – конечное множество и что этот вывод является выводом из множества $P_0$. Применим теорему о полноте еще раз, получим, что формула $F$ является логическим следствием множества $P_0$. $\square$

\subsection{\S 9 Независимость аксиом}


Как мы уже отмечали, есть некоторая математическая традиция, идущая еще от древних греков, согласно которой аксиом в той или иной теории должно быть как можно меньше. В идеальном варианте -- наименее возможное число. Для этого обычно ставится следующий вопрос: будет ли данная аксиома следствием других аксиом? В случае положительно ответа аксиому можно удалить из списка аксиом. В случае отрицательного ответа ее удалить нельзя и аксиома называется \textit{независимой} (от остальных аксиом). Здесь можно сослаться на известную ситуацию в геометрии, а именно на вопрос о том следует ли пятый постулат Евклида из первых четырех. Как показали Лобачевский и Бойяи, пятый постулат независим от остальных аксиом.


В этом параграфе вопрос о независимости мы поставим для схем аксиом в следующей формулировке: можно ли данную схему аксиом получить из остальных схем с помощью прав вывода? Мы решим этот вопрос только для логики высказываний. Так что схемами аксиом будут первые три, а правило вывода будет одно -- модус поненс. Как мы увидим, во всех трех случаях имеет место независимость.

\textbf{Теорема 3.7} Схема $A1$ независима от схем $A2$ и $A3$.

\textit{Доказательство}. Возьмем трехэлементное множество $M = \{0, 1, 2\}$. Элементы этого множества будем обозначать заглавными буквами латиницы. На множестве M введем на нем две операции: унарную $Y = \neg X$ (отрицание) и бинарную $Z = X \rightarrow Y$ (импликацию). Отрицание определяется таблицей 3.1, импликация – таблицей 3.2.

\begin{minipage}{0.5\textwidth}
	\begin{flushleft}
		\begin{tabular}{| c | c |}
			\multicolumn{2}{c}{Таблица 3.1} \\
			\hline
			$X$ & $\neg X$ \\
			\hline
			0 & 1\\
			\hline
			1 & 1\\
			\hline
			2 & 0\\
			\hline
		\end{tabular}
	\end{flushleft}
\end{minipage}
\begin{minipage}{0.5\textwidth}
	\begin{flushright}
		\begin{tabular}{| c | c | c |}
			\multicolumn{3}{c}{Таблица 3.2}\\
			\hline
			$X$ & $Y$ & $X \rightarrow Y$\\
			\hline
			0 & 0 & 0\\
			\hline
			0 & 1 & 2\\
			\hline
			0 & 2 & 0\\
			\hline
			1 & 0 & 2\\
			\hline
			1 & 1 & 2\\
			\hline
			1 & 2 & 0\\
			\hline
			2 & 0 & 0\\
			\hline
			2 & 1 & 0\\
			\hline
			2 & 2 & 0\\
			\hline
		\end{tabular}
	\end{flushright}
\end{minipage}

\textbf{Определение}. Интерпретацией называется функция $\varphi: A \rightarrow \{0, 1, 2\}$, где $A$ --- множество атомарных формул логики высказываний. 

С помощью таблиц 1 и 2 интерпретацию можно расширить на все множество формул логики высказываний.

\textbf{Определение}. Формула $F$ логики высказываний называется выделенной, если для любой интерпретации $\varphi$ выполняется равенство $\varphi(F) = 0$.

Нетрудно проверить, что любая аксиома схемы $A2$ или $A3$ является выделенной. Для аксиомы схемы $A3$ это сделано в таблице 3.3. Для схемы $A2$ соответствующую проверку предлагается сделать читателю.

\begin{tabular}{| c | c | c | c | c | c |}
%	\multicolumn{6}{l}{Таблица 3.3}\\
%	\hline
	$F$ & $G$ & $\neg G \rightarrow \neg F$ & $\neg G \rightarrow F$ & $(\neg G \rightarrow F) \rightarrow G$ & $(\neg G \rightarrow \neg F) \rightarrow [(\neg G \rightarrow F) \rightarrow G]$\\
	\hline
	0 & 0 & 2 & 2 & 0 & 0\\
	\hline
	0 & 1 & 2 & 2 & 0 & 0\\
	\hline
	0 & 2 & 2 & 0 & 0 & 0\\
	\hline
	1 & 0 & 2 & 2 & 0 & 0\\
	\hline
	1 & 1 & 2 & 2 & 0 & 0\\
	\hline
	1 & 2 & 2 & 2 & 0 & 0\\
	\hline
	2 & 0 & 2 & 0 & 0 & 0\\
	\hline
	2 & 1 & 2 & 0 & 2 & 0\\
	\hline
	2 & 2 & 0 & 0 & 2 & 0\\
	\hline
\end{tabular}

\end{document}  